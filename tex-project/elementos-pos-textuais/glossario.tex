\newglossaryentry{ia}{
	name=Inteligência Artificial (IA),
	description={Campo da ciência da computação que se dedica ao desenvolvimento de sistemas capazes de realizar tarefas que normalmente exigiriam inteligência humana, como aprendizado, raciocínio, percepção e compreensão da linguagem natural.}
}

\newglossaryentry{am}{
	name=Aprendizado de Máquina (AM),
	description={Subcampo da Inteligência Artificial que se concentra no desenvolvimento de algoritmos que permitem aos computadores aprender a partir de dados, sem serem explicitamente programados para cada tarefa, melhorando seu desempenho com a experiência.}
}

\newglossaryentry{deep-learning}{
	name=Deep Learning (Aprendizado Profundo),
	description={Subcampo do Aprendizado de Máquina que utiliza redes neurais artificiais com múltiplas camadas (redes neurais profundas) para aprender representações hierárquicas e abstratas dos dados, impulsionando avanços em áreas como visão computacional e PLN.}
}

\newglossaryentry{redes-neurais-artificiais}{
	name=Redes Neurais Artificiais (RNAs),
	description={Modelos computacionais inspirados na estrutura e funcionamento do cérebro biológico, compostos por unidades de processamento interconectadas (neurônios artificiais) que aprendem a partir de dados através de ajustes em suas conexões (pesos).}
}

\newglossaryentry{neuronio-mcculloch-pitts}{
	name=Neurônio de McCulloch-Pitts,
	description={O primeiro modelo de neurônio artificial, proposto em 1943 por Warren McCulloch e Walter Pitts, que estabeleceu uma conexão entre o funcionamento neuronal e a lógica proposicional, lançando as bases para a computação digital e as redes neurais.}
}

\newglossaryentry{retropropagacao}{
	name=Retropropagação do Erro (Backpropagation),
	description={Algoritmo fundamental para o treinamento de redes neurais artificiais, que permite ajustar os pesos das conexões da rede com base no gradiente do erro entre a saída prevista e a saída desejada, propagando esse erro da camada de saída para as camadas anteriores.}
}

\newglossaryentry{transformer}{
	name=Transformer,
	description={Arquitetura de rede neural proposta em 2017, que revolucionou o Processamento de Linguagem Natural ao eliminar a necessidade de recorrência e introduzir o mecanismo de autoatenção, possibilitando o processamento paralelo e a captura eficiente de dependências de longo alcance em sequências.}
}

\newglossaryentry{autoatencao}{
	name=Autoatenção (Self-Attention),
	description={Mecanismo central da arquitetura Transformer que permite a cada elemento de uma sequência (e.g., uma palavra) "pesar" a importância de todos os outros elementos da mesma sequência para determinar sua própria representação contextual.}
}

\newglossaryentry{word-embedding}{
	name=Word Embedding,
	description={Técnica de representação de palavras em um espaço vetorial contínuo e de alta dimensionalidade, onde palavras com significados semelhantes ou que ocorrem em contextos parecidos são mapeadas para vetores numericamente próximos, capturando relações semânticas e sintáticas.}
}

\newglossaryentry{codificacao-posicional}{
	name=Codificação Posicional (Positional Encoding),
	description={Vetores numéricos adicionados aos word embeddings de entrada no Transformer para fornecer informações sobre a posição e ordem dos tokens na sequência, uma vez que o modelo processa elementos em paralelo e não sequencialmente.}
}

\newglossaryentry{pln}{
	name=Processamento de Linguagem Natural (PLN),
	description={Área da Inteligência Artificial dedicada a permitir que computadores compreendam, interpretem e interajam com a linguagem humana de maneira significativa, abrangendo tarefas como análise de sentimento, tradução e geração de texto.}
}

\newglossaryentry{rnn}{
	name=Redes Neurais Recorrentes (RNNs),
	description={Arquitetura de rede neural projetada para processar dados sequenciais, como texto, mantendo uma "memória" ou estado oculto que se propaga ao longo da sequência, permitindo a captura de dependências temporais. Inclui variantes como LSTMs e GRUs.}
}

\newglossaryentry{llm}{
	name=Modelos de Linguagem de Grande Escala (LLMs),
	description={Modelos de linguagem baseados em arquiteturas como o Transformer, treinados em larga escala com quantidades massivas de dados textuais, que exibem capacidades avançadas de generalização, compreensão contextual e geração de texto, como aprendizado zero-shot e few-shot.}
}

\newglossaryentry{aprendizado-autosupervisionado}{
	name=Aprendizado Auto-Supervisionado (Self-Supervised Learning),
	description={Paradigma de treinamento de modelos de IA que gera seus próprios "rótulos" a partir dos dados brutos não rotulados, utilizando tarefas pretexto (e.g., prever palavras mascaradas) para aprender representações complexas, viabilizando o treinamento em escalas massivas.}
}

\newglossaryentry{zero-shot-learning}{
	name=Aprendizado Zero-Shot (Zero-Shot Learning),
	description={Capacidade de um LLM executar uma tarefa para a qual não foi explicitamente treinado, nem viu nenhum exemplo durante o pré-treinamento ou fine-tuning, baseando-se unicamente no vasto conhecimento adquirido.}
}

\newglossaryentry{few-shot-learning}{
	name=Aprendizado Few-Shot (Few-Shot Learning),
	description={Capacidade de um LLM aprender a realizar uma nova tarefa com base em apenas um pequeno número de exemplos (tipicamente de 1 a 5), inferindo o padrão ou a intenção da tarefa a partir do prompt.}
}

\newglossaryentry{quantizacao}{
	name=Quantização,
	description={Técnica de otimização de modelos de IA que reduz a precisão numérica dos parâmetros (e.g., de 32 para 8 bits), diminuindo o consumo de memória e acelerando o tempo de inferência com impacto mínimo na performance.}
}

\newglossaryentry{ollama}{
	name=Ollama,
	description={Plataforma de software de código aberto que simplifica a execução e gerenciamento de Modelos de Linguagem de Grande Escala (LLMs) diretamente em máquinas locais, otimizando o uso de CPUs e GPUs e democratizando o acesso a esses modelos.}
}

\newglossaryentry{ata}{
	name=Ata,
	description={Registro escrito e fiel do que foi tratado, resolvido ou realizado numa reunião, assembleia, sessão, ou evento similar, que após aprovado, assume a condição de documento oficial e pode ter valor legal.}
}

\newglossaryentry{due-diligence}{
	name=Due Diligence,
	description={Processo de investigação e auditoria abrangente e sistemática realizado para analisar e verificar todas as informações relevantes sobre uma parte ou ativo, visando identificar riscos e validar informações antes de fechar um acordo ou tomar uma decisão significativa.}
}

\newglossaryentry{xai}{
	name=Inteligência Artificial Explicável (XAI),
	description={Campo da IA que busca desenvolver métodos para que os sistemas de inteligência artificial não apenas forneçam resultados, mas também justifiquem suas decisões e operem de forma transparente, tornando seu funcionamento compreensível para humanos.}
}



