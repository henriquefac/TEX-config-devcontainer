\chapter{Metodologia}
\label{chap:metodologia}

Esta seção descreve os procedimentos desenvolvidos e adotados neste trabalho, com o objetivo de garantir uma implementação eficiente e assegurar a qualidade 
dos resultados. Inicialmente, apresentam-se os passos relacionados à aquisição, pré-processamento e organização dos dados, visando gerar um \textit{dataset}
adequado para testar a aplicação em suas diferentes fases e validar sua eficácia. Em seguida, detalham-se as etapas de desenvolvimento da aplicação, 
abrangendo sua arquitetura, o fluxo de dados e a integração entre os módulos, com a descrição individual de cada componente.

A metodologia para a definição do \textit{dataset} foi concebida de modo a estabelecer um conjunto de dados capaz de avaliar a qualidade dos resultados 
gerados pela aplicação e validar sua arquitetura. Para tanto, foi desenvolvido um módulo em Python que realiza todas as etapas necessárias à geração dos
dados, desde a aquisição até o refinamento, produzindo um conjunto final que atende aos requisitos mínimos para a validação da aplicação.

Os dados utilizados consistem em tuplas de arquivos PDF e áudios referentes às atas das reuniões. Para possibilitar o processamento e a análise, ambos 
os tipos de arquivo foram convertidos para o formato de texto: os documentos em PDF foram processados por meio de Reconhecimento Óptico de Caracteres (OCR), técnica utilizada para detectar e converter texto presente em imagens em conteúdo textual editável utilizando o Tesseract, enquanto 
os arquivos de áudio foram transcritos com o modelo Whisper. Além disso, foram implementados dois módulos adicionais responsáveis pelo pré-processamento 
dos dados, garantindo a padronização e a limpeza das informações antes da análise.

Após a definição e o pré-processamento do \textit{dataset}, a arquitetura básica do sistema desenvolvido foi projetada para receber transcrições de reuniões 
e gerar documentos PDF, contendo as informações necessárias dentro do modelo estabelecido. O sistema é composto por um \textit{frontend}, um \textit{backend} 
e dois módulos principais: um responsável pela lógica de inteligência artificial e outro dedicado à geração do documento PDF.

\section{Coleta de Dados}
\label{sec:coleta-dados}

Para o teste e a validação da aplicação desenvolvida, foi crucial a formação de um \textit{dataset} completo, bem estruturado e alinhado com a problemática central deste trabalho. Nesse contexto, a etapa inicial concentrou-se na identificação de fontes de dados públicas e confiáveis que pudessem fornecer pares documentais específicos: o texto formal da ata e o registro de como se sucedeu a reunião correspondente, também em texto.

Este conjunto de dados emparelhados é indispensável, não apenas para a validação da aplicação final e a medição de seu desempenho, mas também para a construção do fluxo do sistema em suas fases de processamento. A utilização de dados reais, que espelham o ambiente final de implantação, permitiu o alinhamento e o refinamento dos módulos de aquisição, pré-processamento e análise de forma robusta e representativa.

A instituição selecionada para a extração dos dados foi o Tribunal Regional Eleitoral do Ceará (TRE-CE). Por meio do portal oficial, disponível em \url{https://www.tre-ce.jus.br/servicos-judiciais/sessoes-de-julgamento/sessoes-plenarias}, foi possível acessar a seção de Serviços Judiciários, especificamente a aba de Sessões, Atas e Pautas de Julgamento. Nessa página, encontram-se os registros das sessões do Plenário, abrangendo o período de junho de 2011 até outubro de 2025.

Para os propósitos deste trabalho, optou-se por extrair dados referentes ao intervalo de março de 2020 até outubro de 2025. Os registros disponibilizados no portal consistem em pares de \textit{links} para cada reunião, sendo essa definida através de sua data. Os pares de \textit{links} consistem em: uma URL que leva direto para o documento oficial no formato PDF, e um \textit{link} para a gravação da respectiva reunião, hospedada no YouTube.

O acesso aos dados foi realizado diretamente por meio do portal oficial do TRE-CE, garantindo a confiabilidade e a legitimidade das informações coletadas. Após a identificação das fontes, foi conduzido um processo de averiguação dos dados disponíveis, verificando a integridade e a consistência entre as atas e as gravações. Em seguida, os dados foram classificados de acordo com sua natureza -- documentos PDF e arquivos de áudio.

A complexidade e o volume dos dados a serem processados tornaram impraticável a coleta manual. Para a aquisição em si, foi necessário o desenvolvimento de um módulo de automação em Python capaz de buscar os pares de arquivos dentro do período definido, filtrar os registros válidos, classificá-los baseado na data da reunião e, finalmente, baixar os arquivos para que estivessem prontos para uso no formato esperado pela pipeline que será discutida mais adiante.

Para a implementação desse módulo, foi considerado diversas técnicas, sendo uma delas o \textit{Web Scrapping}. No entanto, visando a robusteza e confiabilidade, foi escolhido uma abordagem mais direta. Investigando o comportamento da página (utilizando ferramentas de desenvolvedor), foi possível deduzir a lógica por trás da construção da URL utilizada para as requisições à API.
\vspace{0.21cm}

O módulo de automação, utilizando a biblioteca \textit{requests} do Python, faz requisições diretas a esses \textit{endpoints} de dados, em vez de processar o HTML da página. Essa abordagem garante que a automação seja mais robusta e resistente a alterações visuais no \textit{frontend} do site, já que ela interage com a camada de dados subjacente. A extração dos metadados em formato estruturado (e.g., JSON) permitiu a classificação e o tratamento rápido das URLs para \textit{download} (PDF) e extração (YouTube). 

Adicionalmente ao módulo principal de coleta, foi desenvolvida uma ferramenta dedicada à interação com os dados, motivada pelo volume potencial de arquivos e pela limitação de espaço de armazenamento disponível. Esta ferramenta opera com uma lógica de acesso \textit{on-demand}: os arquivos são baixados localmente somente quando necessários para o processamento e são automaticamente apagados ao fim da iteração, minimizando o consumo de espaço total.

Contudo, para lidar com pequenas amostras e acelerar o desenvolvimento e testes, o módulo também permite o \textit{download} e o armazenamento permanente dos dados localmente.

Especificamente para os arquivos de áudio e vídeo hospedados no YouTube, foi necessário utilizar a biblioteca \textit{YouTubeFix} para interagir com os canais de \textit{streaming} disponíveis. O módulo foi configurado para acessar e extrair apenas a faixa de áudio na melhor qualidade disponível no formato \textit{webm}.

Com isso, foi estabelecido um canal de acesso em tempo real aos dados brutos de interesse para a aplicação. Todos os próximos passos em relação ao desenvolvimento foram feitos a partir desse ponto, aproveitando a disponibilidade e formatação constante dos dados. Para a construção do \textit{dataset} final, esses dados brutos ainda precisariam passar por uma fase de pré-processamento a fim de padronizar as informações no formato de texto.

\section{Processamento de Dados}
\label{sec:processamento-de-dados}

Esta etapa teve como objetivo a padronização dos dados brutos coletados, ou seja, as atas de reunião em formato PDF e as faixas de áudio no formato \textit{webm}, em uma representação textual estruturada, dequada tanto à aplicação desenvolvida nesse trabalho quanto às técnicas que serão discutidas nos capítulos seguintes, mas mais importante, fiel ao conteúdo original.

A conversão dos dados para formato textual justifica-se, principalmente, por duas razões. Primeiramente, o texto é a modalidade nativa de entrada para a maioria dos modelos de linguagem de larga escala, mesmo que haja, em alguns casos, suporte multimodal. No entanto, mesmo nesses casos, os modelos costumam realizar internamente uma conversão intermediária entre modalidades, por exemplo de áudio para texto, para assim permitir a compreensão semântica do conetúdo, exposto por \citeonline{yin_survey_2023}. Em segundo lugar, como o produto final esperado da pesquisa também é textual, manter o processamento inteiro na mesma modalidade reduz perdas de informação e evita conversões sucessivas, preservando alinhamento semântico entre entrada, processamento e saída.

\subsection{Processamento de arquivos PDF}
\label{sec:processamento-de-arquivos-PDF}

Inicialmente, realizou-se uma análise exploratória dos arquivos em formato PDF com o propósito de identificar a estratégia mais adequada para sua conversão em texto, minimizando perdas de conteúdo ou distorções. Dado a natureza do tipo de arquivo (PDF), foi considerado a extração direta do conteúdo, aproveitando a sua estrutura interna. 

A abordagem inicial utilizou o módulo \textit{PyMuPDF}, responsável por interpretar a estrutura do PDF, extrair o conteúdo textual e armazená-lo em arquivos de texto simples. Entretanto, a inspeção detalhada do conjunto de dados revelou que uma parcela significativa dos arquivos não continha texto embutido, o que indicava que se tratavam de digitalizações de documentos físicos, assim impossibilitando a extração tradicional.

Diante dessa constatação, tornou-se necessária a aplicação de OCR. Assim, o fluxo de processamento passou, então, a iniciar pela detecção do tipo de documento — distinguindo entre PDFs com texto nativo e aqueles compostos por imagens digitalizadas. A classificação define a rota de processamento mais apropriada para cada caso.

Para documentos com texto pesquisável, manteve-se a extração direta via \textit{PyMuPDF}, uma vez que essa abordagem preserva melhor a fidelidade do conteúdo original. Já para os arquivos derivados de material físico, aplicou-se OCR por meio do motor Tesseract. Cada PDF foi previamente segmentado em imagens — uma por página — e o OCR foi executado em paralelo, página a página. Os resultados da conversão, ao final, foram consolidados em arquivos de texto simples, mantendo uma estrutura uniforme para uso nas etapas posteriores do estudo.

\subsection{Processamento de Áudio}
\label{sec:processamento-audio}

A conversão do conteúdo das faixas de áudio para a modalidade textual foi, desde o início, enquadrada como uma tarefa apropriada para modelos de inteligência artificial especializados em reconhecimento automático de fala (ASR — Automatic Speech Recognition). Todos os modelos e ferramentas utilizados são de código aberto e foram executados localmente, assegurando reprodutibilidade, controle total sobre o ambiente de execução e preservação da privacidade dos dados.

Assim como ocorreu com os arquivos em formato PDF, realizou-se inicialmente uma análise exploratória das faixas de áudio extraídas, a fim de compreender suas características e definir um fluxo de processamento adequado ao conjunto de dados.

Durante essa análise, identificaram-se alguns desafios importantes. Um dos primeiros foi a presença de longos períodos de silêncio antes do início e após o término das reuniões, ou seja, períodos que não agragam ao conteúdo final, mas custam no processamento. Além disso, observou-se que as falas dos participantes foram captadas por microfones fixos dispostos sobre as mesas, o que ocasionou variações perceptíveis no volume de voz, uma vez que os interlocutores não mantinham distância constante do microfone.

Outro fator relevante foi a presença de participantes remotos em diversas reuniões. O áudio proveniente dessas intervenções foi registrado de forma indireta — captado pelo som emitido pelos alto-falantes da sala e não diretamente do canal de comunicação digital. Essa dupla captação introduziu eco, reverberação e distorções adicionais, afetando consideravelmente a qualidade do sinal.

Por fim, foi identificado um nível muito alto de ruído no ambiente em quase todos os registros das reuniões, normalmente ocasionado pelo microfone de um participante não estar devidamente desligado quando já havia passado a palavra ou outras fontes externas diversas.

Esses fatores evidenciaram a necessidade de um pré-processamento rigoroso, que se tornou uma etapa fundamental da pipeline. A qualidade do áudio processado impacta diretamente o desempenho do modelo de ASR; portanto, diversas medidas foram adotadas para mitigar os problemas identificados.

A principal ferramenta empregada na etapa de transcrição foi o modelo Whisper, da OpenAI, utilizado em sua implementação otimizada faster-whisper. Esse modelo constitui o núcleo do processo de conversão de fala para texto, e todas as etapas anteriores ao seu uso foram projetadas para maximizar sua precisão.

No pré-processamento, uma das primeiras ações consistiu em converter as faixas de áudio para um formato plenamente compatível com o modelo utilizado. Essa conversão foi realizada com a biblioteca FFmpeg, uma ferramenta amplamente utilizada para manipulação, transcodificação e análise de áudio e vídeo. \\ 

Os arquivos originais, fornecidos em formato WebM, foram reamostrados para uma taxa de 16 kHz, convertidos para canal único (mono) e codificado para o padrão PCM de 16 bits. A adoção desses parâmetros, além de ser o recomendado para ser processado pelo modelo Whisper, visa assegurar a fidelidade do sinal.

Durante o processo de conversão, foram aplicados filtros destinados à melhoria da inteligibilidade do áudio e à redução de heterogeneidades decorrentes das condições reais de gravação. Empregou-se, inicialmente, o filtro dynaudnorm, responsável pela normalização dinâmica do volume, reduzindo oscilações de intensidade que se originam de variações na posição relativa dos interlocutores em relação aos microfones fixos da sala. 

Em complemento, utilizou-se o filtro acompressor, cuja função consiste em atenuar picos abruptos de amplitude e promover maior uniformidade no sinal. A aplicação combinada desses filtros contribuiu para mitigar discrepâncias de volume e aprimorar a clareza acústica do material processado.

O FFmpeg foi configurado para produzir a saída diretamente por meio de fluxo binário (pipe), eliminando a necessidade de criação de arquivos intermediários no sistema de armazenamento. O áudio resultante foi armazenado em um objeto de memória volátil, permitindo seu encaminhamento imediato às etapas subsequentes da pipeline de transcrição. Esse procedimento reduziu o tempo total de processamento, simplificou o fluxo operacional e forneceu ao modelo Whisper um sinal mais consistente e menos suscetível a ruídos e variações indesejadas, favorecendo, assim, a precisão dos resultados obtidos.

Após a conversão e padronização inicial das faixas de áudio, procedeu-se à etapa de redução de ruído, cuja finalidade consistiu em mitigar interferências acústicas presentes nas gravações originais. Para esse fim, empregou-se um modelo de rede neural recorrente (RNN) especialmente treinado para a tarefa de supressão de ruído, disponibilizado por GregorR (2018) em repositório público. O modelo, no formato “.rnnn”, foi executado localmente por meio do filtro arnndn do FFmpeg, que implementa uma interface nativa para modelos de denoising baseados em redes neurais.

O áudio, previamente convertido e normalizado, foi encaminhado ao FFmpeg por meio de fluxo binário, evitando a criação de arquivos intermediários. Durante o processamento, aplicou-se o filtro arnndn, configurado para utilizar o modelo selecionado, o qual é especializado na redução de ruído estacionário e de baixa frequência, característico de ambientes com microfones abertos, ventilação ambiente ou reverberações persistentes. \\

Em complemento, aplicou-se o módulo loudnorm com o objetivo de promover a normalização do nível sonoro percebido, reduzir o intervalo dinâmico e assegurar limites máximos de pico adequados às condições típicas de fala. Essa combinação de filtros resultou em um sinal mais uniforme, com redução significativa de ruídos de fundo e melhora na clareza das vozes registradas.

O resultado do processo foi gerado em formato WAV, com codificação PCM de 16 bits e taxa de amostragem de 16 kHz, preservando a compatibilidade com as etapas subsequentes da pipeline de transcrição. O áudio processado foi mantido em memória volátil e reintegrado ao fluxo operacional sem necessidade de gravação em disco. A etapa de redução de ruído mostrou-se especialmente relevante devido à presença recorrente de interferências ambientais nas gravações das reuniões, impactando diretamente a acurácia do modelo de ASR utilizado.


\section{Métricas de Avaliação}
\label{sec:metrica-de-avaliacao}